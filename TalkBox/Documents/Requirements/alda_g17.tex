%%%%%%%%%%%%%%%%%%%%%%%%%%%%%%%%%%%%%%%%%%%%%%%%%%%%%%%%%%%%%%%%%%%%%%%%%%%%%%%%%
%% Documentclass 
%%%%%%%%%%%%%%%%%%%%%%%%%%%%%%%%%%%%%%%%%%%%%%%%%%%%%%%%%%%%%%%%%%%%%%%%%%%%%%%%%
\documentclass[a4paper,oneside,titlepage]{report}
%%%%%%%%%%%%%%%%%%%%%%%%%%%%%%%%%%%%%%%%%%%%%%%%%%%%%%%%%%%%%%%%%%%%%%%%%%%%%%%%%
%% Packages
%%%%%%%%%%%%%%%%%%%%%%%%%%%%%%%%%%%%%%%%%%%%%%%%%%%%%%%%%%%%%%%%%%%%%%%%%%%%%%%%%
\usepackage[english]{babel}
\usepackage{amsmath}
\usepackage{complexity}
\usepackage[T1]{fontenc}
\usepackage[utf8]{inputenc}
\usepackage[pdftex]{graphicx} %%Graphics in pdfLaTeX
\usepackage{a4wide} %%Smaller margins, more text per page.
\usepackage{longtable} %%For tables that exceed a page width
\usepackage{pdflscape} %%Adds PDF support to the landscape environment of package
\usepackage{caption} %%Provides many ways to customise the captions in floating environments like figure and table
\usepackage{float} %%Improves the interface for defining floating objects such as figures and tables
\usepackage[tablegrid,nochapter]{vhistory} %%Vhistory simplifies the creation of a history of versions of a document
\usepackage[nottoc]{tocbibind} %%Automatically adds the bibliography and/or the index and/or the contents, etc., to the Table of Contents listing
\usepackage[toc,page]{appendix} %%The appendix package provides various ways of formatting the titles of appendices
\usepackage{pdfpages} %%This package simplifies the inclusion of external multi-page PDF documents in LATEX documents
\usepackage[rightcaption]{sidecap} %%Defines environments called SCfigure and SCtable (analogous to figure and table) to typeset captions sideways
\usepackage{cite} %%The package supports compressed, sorted lists of numerical citations, and also deals with various punctuation and other issues of representation, including comprehensive management of break points
\usepackage[]{acronym} %%This package ensures that all acronyms used in the text are spelled out in full at least once. It also provides an environment to build a list of acronyms used
\usepackage[pdftex,scale={.8,.8}]{geometry} %%The package provides an easy and flexible user interface to customize page layout, implementing auto-centering and auto-balancing mechanisms so that the users have only to give the least description for the page layout. For example, if you want to set each margin 2cm without header space, what you need is just \usepackage[margin=2cm,nohead]{geometry}.
\usepackage{layout} %%The package defines a command \layout, which will show a summary of the layout of the current document
\usepackage{subfigure} %%Provides support for the manipulation and reference of small or ‘sub’ figures and tables within a single figure or table environment.
\usepackage[toc]{glossaries} %%The glossaries package supports acronyms and multiple glossaries, and has provision for operation in several languages (using the facilities of either babel or polyglossia).
\usepackage[left,pagewise,modulo]{lineno} %%Adds line numbers to selected paragraphs with reference possible through the LATEX \ref and \pageref cross reference mechanism
\usepackage[pdftex,colorlinks=false,hidelinks,pdfstartview=FitV]{hyperref}%%The hyperref package is used to handle cross-referencing commands in LATEX to produce hypertext links in the document. 
\usepackage{metainfo}
\usepackage[pagestyles,raggedright]{titlesec}
\usepackage{etoolbox}
\usepackage{%
	array, %%An extended implementation of the array and tabular environments which extends the options for column formats, and provides "programmable" format specifications
	booktabs, %%The package enhances the quality of tables in LATEX, providing extra commands as well as behind-the-scenes optimisation
	dcolumn, %%
	rotating,
	shortvrb,
	units,
	url,
	lastpage,
	longtable,
	lscape,
	qtree,
	skmath,	
}
%%%%%%%%%%%%%%%%%%%%%%%%%%%%%%%%%%%%%%%%%%%%%%%%%%%%%%%%%%%%%%%%%%%%%%%%%%%%%%%%%
%% Java --> latex 
%%%%%%%%%%%%%%%%%%%%%%%%%%%%%%%%%%%%%%%%%%%%%%%%%%%%%%%%%%%%%%%%%%%%%%%%%%%%%%%%%
\usepackage{listings}
\usepackage{color}
\definecolor{pblue}{rgb}{0.13,0.13,1}
\definecolor{pgreen}{rgb}{0,0.5,0}
\definecolor{pred}{rgb}{0.9,0,0}
\definecolor{pgrey}{rgb}{0.46,0.45,0.48}
\usepackage{inconsolata}
%%Listing style for java.
\definecolor{dkgreen}{rgb}{0,0.6,0}
\definecolor{gray}{rgb}{0.5,0.5,0.5}
\definecolor{mauve}{rgb}{0.58,0,0.82}
\lstset{frame=tb,
	language=Java,
	aboveskip=3mm,
	belowskip=3mm,
	showstringspaces=false,
	columns=flexible,
	basicstyle={\small\ttfamily},
	numbers=left,
	numberstyle=\tiny\color{gray},
	keywordstyle=\color{blue},
	commentstyle=\color{dkgreen},
	stringstyle=\color{mauve},
	breaklines=true,
	breakatwhitespace=true,
	tabsize=3
}

%%%%%%%%%%%%%%%%%%%%%%%%%%%%%%%%%%%%%%%%%%%%%%%%%%%%%%%%%%%%%%%%%%%%%%%%%%%%%%%%%
\setlength{\parindent}{0pt}
\setlength{\parskip}{.5\baselineskip}
%%%%%%%%%%%%%%%%%%%%%%%%%%%%%%%%%%%%%%%%%%%%%%%%%%%%%%%%%%%%%%%%%%%%%%%%%%%%%%%%%
%% Inserting the metadata
%%%%%%%%%%%%%%%%%%%%%%%%%%%%%%%%%%%%%%%%%%%%%%%%%%%%%%%%%%%%%%%%%%%%%%%%%%%%%%%%%
% % Document Metadata

\def\Company{Group 15}
\def\Institute{\textit{York University}}
\def\Course{\textit{EECS2311 Software Development Project}}

\def\BoldTitle{Software Requirements Specification}

\def\Subtitle{for \\ TalkBox Software System \\}
\def\Authors{Prepared by Manish Bhasin} 
\def\Shortname{A.Sandu}


\title{\textbf{\BoldTitle}\\\Subtitle}
\author{\Authors \\ \\ \\ \Institute\\ \Course\\ }
\date{03. February 2019}

%%%%%%%%%%%%%%%%%%%%%%%%%%%%%%%%%%%%%%%%%%%%%%%%%%%%%%%%%%%%%%%%%%%%%%%%%%%%%%%%%
%% Creation of pdf information
%%%%%%%%%%%%%%%%%%%%%%%%%%%%%%%%%%%%%%%%%%%%%%%%%%%%%%%%%%%%%%%%%%%%%%%%%%%%%%%%%
\hypersetup{pdfinfo={
		Title={Title},
		Author={TR},
		Subject={Report}
	}}
%%%%%%%%%%%%%%%%%%%%%%%%%%%%%%%%%%%%%%%%%%%%%%%%%%%%%%%%%%%%%%%%%%%%%%%%%%%%%%%%%
%% Creating the frontpage
%%%%%%%%%%%%%%%%%%%%%%%%%%%%%%%%%%%%%%%%%%%%%%%%%%%%%%%%%%%%%%%%%%%%%%%%%%%%%%%%%
\AtBeginDocument{
	\maketitle
	\thispagestyle{empty}
}

%%%%%%%%%%%%%%%%%%%%%%%%%%%%%%%%%%%%%%%%%%%%%%%%%%%%%%%%%%%%%%%%%%%%%%%%%%%%%%%%%
%% Creation of the header
%%%%%%%%%%%%%%%%%%%%%%%%%%%%%%%%%%%%%%%%%%%%%%%%%%%%%%%%%%%%%%%%%%%%%%%%%%%%%%%%%
\patchcmd{\chapter}{plain}{short}{}{} %$ <-- the header on chapter 1
%%%%%%%%%%%%%%%%%%%%%%%%%%%%%%%%%%%%%%%%%%%%%%%%%%%%%%%%%%%%%%%%%%%%%%%%%%%%%%%%%
%% Creation of page-styles
%%%%%%%%%%%%%%%%%%%%%%%%%%%%%%%%%%%%%%%%%%%%%%%%%%%%%%%%%%%%%%%%%%%%%%%%%%%%%%%%%
\newpagestyle{long}{%
	\sethead[\thepage][][\chaptername\ \thechapter:\ \chaptertitle]{\chaptername\ \thechapter:\ \chaptertitle}{}{\thepage}
	\headrule
}

\newpagestyle{short}{%
	\sethead[\thepage][][]{}{}{\thepage}
	\headrule
}
%%%%%%%%%%%%%%%%%%%%%%%%%%%%%%%%%%%%%%%%%%%%%%%%%%%%%%%%%%%%%%%%%%%%%%%%%%%%%%%%%
%% DOCUMENT
%%%%%%%%%%%%%%%%%%%%%%%%%%%%%%%%%%%%%%%%%%%%%%%%%%%%%%%%%%%%%%%%%%%%%%%%%%%%%%%%%
\begin{document}

\pagenumbering{roman}
\DeclareGraphicsExtensions{.pdf,.jpg,.png}
\pagestyle{short}



\newpage
%%%%%%%%%%%%%%%%%%%%%%%%%%%%%%%%%%%%%%%%%%%%%%%%%%%%%%%%%%%%%%%%%%%%%%%%%%%%%%%%%
%% Table of contents
%%%%%%%%%%%%%%%%%%%%%%%%%%%%%%%%%%%%%%%%%%%%%%%%%%%%%%%%%%%%%%%%%%%%%%%%%%%%%%%%%
 \tableofcontents % Inhaltsverzeichnis



\pagestyle{long}




%%%%%%%%%%%%%%%%%%%%%%%%%%%%%%%%%%%%%%%%%%%%%%%%%%%%%%%%%%%%%%%%%%%%%%%%%%%%%%%%%
%% Version table insertion
%%%%%%%%%%%%%%%%%%%%%%%%%%%%%%%%%%%%%%%%%%%%%%%%%%%%%%%%%%%%%%%%%%%%%%%%%%%%%%%%%
% Versionstabelle.

\chapter*{Revision History}
\addcontentsline{toc}{chapter}{Revision History}
\begin{versionhistory}
	\vhEntry{1.0}{04.02.2019}{M.Qureshi}{Chapter 1 - Introduction}
    \vhEntry{2.0}{04.02.2019}{M.Qureshi}{Chapter 3 - Testing Checklist}

\end{versionhistory}
\pagenumbering{arabic}
%%%%%%%%%%%%%%%%%%%%%%%%%%%%%%%%%%%%%%%%%%%%%%%%%%%%%%%%%%%%%%%%%%%%%%%%%%%%%%%%%
%% Inserting all the content
%%%%%%%%%%%%%%%%%%%%%%%%%%%%%%%%%%%%%%%%%%%%%%%%%%%%%%%%%%%%%%%%%%%%%%%%%%%%%%%%%
\chapter{Introduction}
\label{ch:intro}
\section{Purpose}
This document provides information on test cases for the TalkBox application. This document covers the several test cases the application has, as well as justified derivation of each of these test cases. Test case derivation is justified thoroughly for both the simulator component of the TalkBox, as well as the configuration application component. The sufficiency of each test case is provided through this document as well alongside test coverage.
\section{Testing Checklist}
There are two testing checklists in the Testing Checklist chapter, one for the TalkBox configuration application and another for the TalkBox simulator application. These checklists are broken into two subsections of the chapter.
\section{Test Case Derivation}
In the Test Case Derivation chapter of this document, a through description of how each testcase was derived is provided. There are two subsections to the Test Case Derivation chapter, one for the configuration application and another for the simulator application. In each of these subsections the respective applications test case derivation is provided. 
\section{Test Case Sufficiency}
In the Test Case Sufficiency chapter of this document, a justification for each test case from the Test Case Derivation chapter is provided. There are two subsections to the Test Case Sufficiency chapter, one for the configuration application and another for the simulator application. In each of these subsections the respective applications test cases are justified with proven sufficiency.
\section{Test Case Implementation}

In the Test Case Implementation section of this document, a description of how each test case is implemented is provided. There are two subsections to the Test Case Implementation chapter, one for the configuration application and another for the simulator application. In each of these subsections, the respective applications test case implementations are shown and justified.


\section{Test Coverage}

In the Test Coverage chapter of this document, ….


\chapter{Overall Description}
\label{Overall Description}

\section{Testing Documents Description}

Throughout this document there are several chapters that show different test cases as well as the respective deviation, sufficiency, and implementation of each of these test cases. A test coverage is also provided in this document. An understanding of the testing done and developed during the making of the TalkBox are thoroughly explored and conveyed throughout this document. 

\newpage

\chapter{Testing Checklist}
\label{External Interface Requirements}

\section{TalkBox Configuration Application}
\section{TalkBox Simulator Application}

\chapter{Test Case Derivation}
\label{System Features}

\section{How Test Cases Were Derived: TalkBox Configuration Application}
\section{How Test Cases Were Derived: TalkBox Simulator Application}

\chapter{Test Case Sufficiency}
\label{Other Nonfunctional Requirements}

\section{Why Test Cases Are Sufficient}


\chapter{Test Case Implementation}
\label{Other Requirements}
\section{How Are Test Cases Implemented}






%%%%%%%%%%%%%%%%%%%%%%%%%%%%%%%%%%%%%%%%%%%%%%%%%%%%%%%%%%%%%%%%%%%%%%%%%%%%%%%%%
%% Source defintions
%%%%%%%%%%%%%%%%%%%%%%%%%%%%%%%%%%%%%%%%%%%%%%%%%%%%%%%%%%%%%%%%%%%%%%%%%%%%%%%%%
% When no use outcomment
%\bibliographystyle{alpha}

\renewcommand\bibname{References}
\bibliography{base/sources}


%%%%%%%%%%%%%%%%%%%%%%%%%%%%%%%%%%%%%%%%%%%%%%%%%%%%%%%%%%%%%%%%%%%%%%%%%%%%%%%%%
%% Inserting the appendix
%%%%%%%%%%%%%%%%%%%%%%%%%%%%%%%%%%%%%%%%%%%%%%%%%%%%%%%%%%%%%%%%%%%%%%%%%%%%%%%%%
% When no use outcomment
%\vspace{1cm}

% % Wenn nicht im Inhaltsverzeichnis stehen soll:
\chapter*{Acronyms}
\addcontentsline{toc}{chapter}{Acronyms}

% % Wenns im Inhaltsverzeichnis stehen soll:
%\chapter{Liste der Abkürzungen}

% % Abkürzungen
% Das "ABCD" steht synonym für die längste Abkürzung in der Liste. Falls eine längere auftaucht, bitte anpassen.
\begin{acronym}[ABCD]

\end{acronym}

\end{document}*/***********************************************************************8	
